\documentclass[10pt,a4paper]{article}

% Fonts
\usepackage{fontspec}
% \setmainfont{Palatino Linotype}
\setsansfont{Arial}

% Page margins
\usepackage[a4paper,margin=1in]{geometry}

% Line spacing
\usepackage{setspace}
\onehalfspacing

% Justified composition
\usepackage{ragged2e}
\justifying

% Other packages
\usepackage{hyperref}
\usepackage{authblk}

\begin{document}

\title{Prototyping ``ReGroup'' mobile application}
\author[1]{Simone Ruberto}
\author[1]{Marco Rotiroti}
\author[1]{Riccardo Ruberto}
\author[1]{Vladimiro Paschali}
\author[1]{Gianluca Vannoli}
\affil[1]{Sapienza University of Rome}

\date{\today}

\maketitle

\section*{Create a profile picture}
Suppose you are in a museum with other participants. \\
The visit will start in a few minutes, and your teacher will create a new group. \\
You must be ready to join the group by setting up your profile picture.
\subsection*{Version 1}
\begin{itemize}
    \item \textbf{Tests and Problems}
    \begin{itemize}
    \item The incomprehensible interface button and the profile and settings screen confused users about the right one to set up the image.
    \end{itemize}
    \item \textbf{Resolution}
    \begin{itemize}
        \item Merge the two screens into the profile screen only and change its icon into a self-explanatory one.
    \end{itemize}
\end{itemize}

\subsection*{Version 2}
\begin{itemize}
    \item \textbf{Tests and Problems}
    \begin{itemize}
        \item The profile icon was too small.  
    \end{itemize}
    \item \textbf{Positive Aspects}
    \begin{itemize}
        \item The profile icon understood when seen.
    \end{itemize}    
    \item \textbf{Resolution}
    \begin{itemize}
        \item Resize the profile icon
        \item Lighten screens by changing the colour and making them more minimalist.
    \end{itemize}
\end{itemize}

\newpage
\section*{Create a group}
Suppose you want to create a group. \\
You want to hide the profile picture of the group's members.
\subsection*{Version 1}
\begin{itemize}
    \item \textbf{Tests and Problems}
    \begin{itemize}
        \item Difficulty in clicking the button for creating the group.
        \item Lack of a setting to show/hide participants' photos.
        \item Users are confused by the non-clickable GPS button.
    \end{itemize}
    \item \textbf{Resolution}
    \begin{itemize}
        \item Replace the big Material Design card with a button.    
        \item Introduce an option to toggle the visibility of participant photos in the group creation process.
        \item Simplify the user experience by excluding the step that involves selecting between Bluetooth and GPS technology.
    \end{itemize}
\end{itemize}

\subsection*{Version 2}
\begin{itemize}
    \item \textbf{Tests and Problems}
    \begin{itemize}
        \item Oversized group creation button.
        \item Complex screen layout and too bright colours.
    \end{itemize}
    \item \textbf{Resolution}
    \begin{itemize}
        \item Reduce the size of the group creation button to enhance usability.
        \item Improve user understanding and ease of navigation by simplifying the layout of the screen.
    \end{itemize}
\end{itemize}

\newpage
\section*{Join a group}
Suppose your teacher created a group. \\
She provided the information to join the group. \\
You want to join the group in the shortest possible time.
\subsection*{Version 1}
\begin{itemize}
    \item \textbf{Tests and Problems}
    \begin{itemize}
        \item Group identifier too long.
        \item Users are confused by the non-clickable GPS button.
    \end{itemize}
    \item \textbf{Resolution}
    \begin{itemize}
        \item Implemented a new algorithm for generating the group identifier.
        \item Removed the step that involves clicking on the GPS button.                
    \end{itemize}
\end{itemize}

\subsection*{Version 2}
\begin{itemize}
    \item \textbf{Tests and Problems}
    \begin{itemize}
        \item The group identifier still needs to be entered by the user manually.        
        \item Low contrast between the background color and other components on the screen.
    \end{itemize}
    \item \textbf{Positive Aspects}
    \begin{itemize}
        \item Reduced length of the group identifier.
        \item Shortened the duration for entering the group identifier manually.
    \end{itemize}
    \item \textbf{Resolution}
    \begin{itemize}
        \item Replaced the screen background colour.
        \item Group identifiers can now be manually entered or scanned using a QR Code.
    \end{itemize}
\end{itemize}

\end{document}
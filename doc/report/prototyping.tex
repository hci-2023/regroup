\section{Prototyping}
\subsection{Scenarios}
\begin{itemize}
    \item \textbf{Create a profile picture:} Suppose you are in a museum with other participants. The visit will start in a few minutes, and your teacher will create a new group. You must be ready to join the group by setting up your profile picture.
    \item \textbf{Create a group:} Suppose you want to create a group. You want to hide the profile picture of the group's members.
    \item \textbf{Join a group:} Suppose your teacher created a group. She provided the information to join the group. You want to join the group in the shortest possible time.
\end{itemize}

\subsection{Used Approach}
Testing involved groups of 3-4 university students, with an evolutionary prototyping approach employed for each task.  With each iteration, modifications were suggested to enhance the prototype and address issues observed in the previous version. The evaluation method employed was Think Aloud.

\subsection{Iterations and Changes}
For details regarding the positive and negative aspects encountered in different versions and the corresponding solutions to previous issues, please consult the document \textbf{prototyping.pdf}.
